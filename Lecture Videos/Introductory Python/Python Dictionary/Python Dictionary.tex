\documentclass{beamer}

% You can uncomment the themes below if you would like to use a different
% one:
%\usetheme{AnnArbor}
%\usetheme{Antibes}
%\usetheme{Bergen}
% \usetheme{Berkeley}
%\usetheme{Berlin}
%\usetheme{Boadilla}
%\usetheme{boxes}
%\usetheme{CambridgeUS}
%\usetheme{Copenhagen}
%\usetheme{Darmstadt}
%\usetheme{default}
%\usetheme{Frankfurt}
%\usetheme{Goettingen}
%\usetheme{Hannover}
%\usetheme{Ilmenau}
%\usetheme{JuanLesPins}
%\usetheme{Luebeck}
\usetheme{Madrid}
%\usetheme{Malmoe}
%\usetheme{Marburg}
%\usetheme{Montpellier}
%\usetheme{PaloAlto}
%\usetheme{Pittsburgh}
%\usetheme{Rochester}
%\usetheme{Singapore}
%\usetheme{Szeged}
% \usetheme{Warsaw}

\usepackage{kotex} % For using korean. Do not modify this line.
\usepackage{listings} % For embedding codes. Do not modifiy this line. 

\usepackage{datetime} % For using date at compile time. 

\logo{{\includegraphics[height=0.7cm]{../../../References/Template Images/fastcampus-logo-positive.png}}}

% slide environment 
\newenvironment{slide}[1][]
{%
  \begin{frame}[allowframebreaks,#1]%
  }{%
  \end{frame}%
}
% \newcounter{totalcontinuationcount}
% \makeatletter
% \setbeamertemplate{frametitle continuation}{%
    % \setcounter{totalcontinuationcount}{\beamer@endpageofframe}%
    % \addtocounter{totalcontinuationcount}{1}%
    % \addtocounter{totalcontinuationcount}{-\beamer@startpageofframe}%
    % \ifnum \value{totalcontinuationcount} > 1
        % \textmd{(\insertcontinuationcount/\arabic{totalcontinuationcount})}%
    % \fi
% }
% \makeatother
\setbeamertemplate{frametitle continuation}[from second][\insertcontinuationcountroman]

% Code embeddings 

\renewcommand{\lstlistingname}{Code}
\renewcommand{\lstlistlistingname}{List of \lstlistingname s}

\newcommand{\includecode}[3]{\lstinputlisting[caption=#3, style=#1]{#2}} % For code inclusion. 

\newcommand{\python}[2]{\lstinputlisting[caption=#2, style=python]{#1}} % For code inclusion. 

\newcommand{\cpp}[2]{\lstinputlisting[caption=#2, style=cpp]{#1}} % For code inclusion. 


\usepackage{color}
\definecolor{dkgreen}{rgb}{0,0.6,0}
\definecolor{gray}{rgb}{0.5,0.5,0.5}
\definecolor{mauve}{rgb}{0.58,0,0.82}

\lstdefinestyle{python}{frame=tb,
  language=Python,
  aboveskip=3mm,
  belowskip=3mm,
  showstringspaces=false,
  columns=flexible,
  basicstyle={\small\ttfamily},
  numbers=left,
  numberstyle=\tiny\color{gray},
  keywordstyle=\color{blue},
  commentstyle=\color{dkgreen},
  stringstyle=\color{mauve},
  breaklines=true,
  breakatwhitespace=true,
  tabsize=4
}

\lstdefinestyle{cpp}{frame=tb,
  language=C++,
  aboveskip=3mm,
  belowskip=3mm,
  showstringspaces=false,
  columns=flexible,
  basicstyle={\small\ttfamily},
  numbers=left,
  numberstyle=\tiny\color{gray},
  keywordstyle=\color{blue},
  commentstyle=\color{dkgreen},
  stringstyle=\color{mauve},
  breaklines=true,
  breakatwhitespace=true,
  tabsize=4
}


\title{파이썬 Dictionary}

% A subtitle is optional and this may be deleted
\subtitle{Fastcampus Online Lecture}

\author{신승우}


% Let's get started
\begin{document}

\begin{slide}
  \titlepage
\end{slide}

\begin{slide}{Outline}
  \tableofcontents %[hideallsubsections]
  % You might wish to add the option [pausesections]
\end{slide}

% Section and subsections will appear in the presentation overview
% and table of contents.

\section{Dictionary 소개} 

\begin{slide}[fragile, environment=slide]{Why Dictionary?} 
% 영어 단어와 그에 대응되는 한국어 단어의 순서쌍이 있다고 합시다. 앞에서 배운 자료형을 이용하면 다음과 같이 나타낼 수 있을 것입니다. 
\python{eng_kor.py}{영-한 단어 리스트}
% 이 때, 한국어 단어에 대응되는 한국어 단어를 찾고 싶다고 합시다. 예를 들어서, 사과를 입력받으면 apple을 반환하는 함수를 짜고 싶다고 합시다. 다음과 같이 짤 수 있을 것입니다. 
\framebreak
\python{translate_list.py}{영-한 단어}
% 이 함수는 정확한 결과를 리턴하지만, 매번 저런 식의 함수를 짜는 것은 번거롭습니다. 이러한 문제를 해결하기 위해서 파이썬에서는 사전 자료형을 제공합니다. 
\end{slide}

\begin{slide}[fragile, environment=slide]{파이썬 Dictionary} 
\python{eng_kor.py}{영-한 사전}
% 앞 슬라이드에서 예시로 나온 한-영 단어들을 파이썬 사전을 이용해서는 다음과 같이 나타낼 수 있습니다. 사전 내의 원소들은 ,으로 나눠지며, :을 사이에 두고 있어야만 합니다. : 앞의 원소를 key, 뒤의 원소를 그 key의 값에 해당하는 value라고 합니다. 예를 들어서, 저 사전에서 사과는 key이고, apple은 그 key에 해당하는 value입니다. 
\end{slide}

\section{Dictionary에서 쓸 수 있는 함수들} 

\begin{slide}[fragile, environment=slide]{Dictionary - 원소들 다루기} 
\python{get_item.py}{사전형 다루기}
% 사전에서 대응되는 값들을 다음과 같이 다룰 수 있습니다. 
\end{slide}



\begin{slide}[fragile, environment=slide]{Dictionary - for loop} 
\python{iter_dict.py}{사전형 다루기}
% 사전 안 원소들을 다음과 같이 for loop를 이용하여 순회할 수 있습니다. 
\end{slide}

\begin{slide}[fragile, environment=slide]{예제 - 사전 만들기} 
\begin{table}[인적사항]
\begin{tabular}{|l|l|}
\hline
이름 & 신승우         \\ \hline
id & principia12 \\ \hline
생일 & 1992.12.19  \\ \hline
\end{tabular}
\end{table}
위 표의 내용을 담고 있는 dictionary를 만들어보세요. 
\end{slide}


\begin{slide}[fragile, environment=slide]{예제 - 사전 뒤집기} 
\python{example.py}{사전 뒤집기}
위 코드가 동작하도록 reverse\_dict 함수를 구현해보세요. 
% 사전 안 원소들을 다음과 같이 for loop를 이용하여 순회할 수 있습니다. 
\end{slide}


% \section{Dictionary 더 살펴보기}

% \begin{slide}[fragile, environment=slide]{Dictionary - unhashable type} 
% \end{slide}

% \begin{slide}[fragile, environment=slide]{Dictionary - mutable types} 
% \end{slide}

\section{Dictionary의 용도} 

\begin{slide}[fragile, environment=slide]{Dictionary의 용도 예시  -  Json} 
% 위에서 사전형 데이터를 살펴보았습니다. 이런 사전형 데이터는 복잡한 데이터를 나타낼 때 용이한데, 이는 리스트나 튜플과는 다르게 데이터에 임의의 색인을 부여할 수 있기 때문입니다. 예를 들어서, 리스트에서는 각 원소들에 0부터 시작하는 index가 부여됩니다. 사전은 이를 일반화하여, 원하는 색인을 자유롭게 부여할 수 있는 데이터형이라고 볼 수 있습니다. 따라서 사전형은 복잡한 데이터를 나타낼 때 용이합니다. 예를 들어서, 위 예제 1에서 나왔던 것처럼 인적사항 같은 데이터를 나타내기에 용이합니다. 여기서는 json을 예시로 들어 살펴보겠습니다. 

Json이란? 
\begin{itemize} 
\item JavaScript Object Notation 
\item 웹에서 정보를 주고 받을 때 사용 
\end{itemize}

\end{slide}


\begin{slide}[fragile, environment=slide]{Dictionary의 용도 예시  -  Json} 
% 네이버 스포츠면을 예시로 살펴보겠습니다. 네이버 스포츠 기사 url을 잘 보면, url = 'https://sports.news.naver.com/kbaseball/news/index.nhn?page=20&date=20160703&isphoto=N' 
% 처럼 url 뒤에 ?로 된 부분이 있고, 이 부분 뒤에 date 같은 부분이 있습니다. 이 부분은 네이버에 우리가 어느 날의 기사를 보고 싶은지를 알려줍니다. 이 때, 주고받는 데이터를 파이썬에서는 dict를 이용하여 json을 만들고, 이것을 이용하여 네이버에게 필요한 데이터를 주거나 받게 됩니다.

\python{request_example.py}{사전 뒤집기}
위 예시를 사용하기 위해서는 pip install requests를 이용하여 requests 모듈을 다운로드 받으셔야 합니다. 

\end{slide}



\end{document}


