\documentclass[11pt, a4paper]{article}
%\usepackage{geometry}
\usepackage[inner=2.5cm,outer=2.5cm,top=2.5cm,bottom=2.5cm, textwidth=5cm]{geometry}
\pagestyle{empty}
\usepackage{graphicx}
\usepackage{fancyhdr, lastpage, bbding, pmboxdraw}
\usepackage[usenames,dvipsnames]{color}
\definecolor{darkblue}{rgb}{0,0,.6}
\definecolor{darkred}{rgb}{.7,0,0}
\definecolor{darkgreen}{rgb}{0,.6,0}
\definecolor{red}{rgb}{.98,0,0}
\usepackage[colorlinks,pagebackref,pdfusetitle,urlcolor=darkblue,citecolor=darkblue,linkcolor=darkred,bookmarksnumbered,plainpages=false]{hyperref}
\renewcommand{\thefootnote}{\fnsymbol{footnote}}



\pagestyle{fancyplain}
\fancyhf{}
\lhead{ \fancyplain{}{MAS101} }
%\chead{ \fancyplain{}{} }
\rhead{ \fancyplain{}{\today} }
%\rfoot{\fancyplain{}{page \thepage\ of \pageref{LastPage}}}
\fancyfoot[RO, LE] {page \thepage\ of \pageref{LastPage} }
\thispagestyle{plain}

%%%%%%%%%%%% LISTING %%%
\usepackage{listings}
\usepackage{caption}
\DeclareCaptionFont{white}{\color{white}}
\DeclareCaptionFormat{listing}{\colorbox{gray}{\parbox{\textwidth}{#1#2#3}}}
\captionsetup[lstlisting]{format=listing,labelfont=white,textfont=white}
\usepackage{verbatim} % used to display code
\usepackage{fancyvrb}
\usepackage{acronym}
\usepackage{kotex}
\usepackage{amsthm}
\VerbatimFootnotes % Required, otherwise verbatim does not work in footnotes!



\definecolor{OliveGreen}{cmyk}{0.64,0,0.95,0.40}
\definecolor{CadetBlue}{cmyk}{0.62,0.57,0.23,0}
\definecolor{lightlightgray}{gray}{0.93}



\lstset{
%language=bash,                          % Code langugage
basicstyle=\ttfamily,                   % Code font, Examples: \footnotesize, \ttfamily
keywordstyle=\color{OliveGreen},        % Keywords font ('*' = uppercase)
commentstyle=\color{gray},              % Comments font
numbers=left,                           % Line nums position
numberstyle=\tiny,                      % Line-numbers fonts
stepnumber=1,                           % Step between two line-numbers
numbersep=5pt,                          % How far are line-numbers from code
backgroundcolor=\color{lightlightgray}, % Choose background color
frame=none,                             % A frame around the code
tabsize=2,                              % Default tab size
captionpos=t,                           % Caption-position = bottom
breaklines=true,                        % Automatic line breaking?
breakatwhitespace=false,                % Automatic breaks only at whitespace?
showspaces=false,                       % Dont make spaces visible
showtabs=false,                         % Dont make tabls visible
columns=flexible,                       % Column format
morekeywords={__global__, __device__},  % CUDA specific keywords
}

%%%%%%%%%%%%%%%%%%%%%%%%%%%%%%%%%%%%
\begin{document}
\begin{center}
{\Large \textsc{MAS101 : Coding the Mathematics - an OOP approach}}
\end{center}
\begin{center}
\today
\end{center}
%\date{September 26, 2014}

\begin{center}
\rule{6in}{0.4pt}
\begin{minipage}[t]{.75\textwidth}
\begin{tabular}{llcccll}
\textbf{Lecturer:} & Seungwoo Schin & & &  & \textbf{Time:} & TBA \\
\textbf{Email:} &  \href{mailto:principia_12@kaist.ac.kr}{principia\_12@kaist.ac.kr} & & & & \textbf{Place:} & TBA
\end{tabular}
\end{minipage}
\rule{6in}{0.4pt}
\end{center}
\vspace{.5cm}
\setlength{\unitlength}{1in}
\renewcommand{\arraystretch}{2}


\noindent\textbf{Study Objectives} 머신 러닝에 필요한 수학적인 개념을 직접 파이썬으로 구현해보면서 익히는 것을 목표로 합니다. 


\vskip.15in
\noindent\textbf{Target Audience} 
\begin{itemize}
\item 선형대수학과 미적분학, 확률 및 통계의 기본적인 개념이 궁금하신 분 
\item 다양한 코딩 기술을 다지고 싶으신 분 
\end{itemize}


\vskip.15in
\noindent\textbf{Things to Learn} 
\begin{itemize}
\item 벡터, 행렬에 대한 기본적인 개념 및 연산 
\item 선형대수학을 활용한 문제풀이 
\item 수학 문제 적용에 대한 예시 찾아보기
\end{itemize}

\vskip.15in
\noindent\textbf{Policy} 
\begin{itemize} 
\item 본 강의는 수학 이론을 배우고, 배운 이론을 구현해보는 것으로 이루어져 있습니다. 
\item 실습 코드는 개설될 Github Repository에 커밋하여 공유하게 할 예정입니다. 
\item 실습을 하는 것이 의무는 아니지만, 가능하면 실습 문제를 해결해볼 것을 권장드립니다. 
\end{itemize}


\vskip.15in
\noindent\textbf{Prerequisite} : 기본적인 파이썬 문법에 대한 지식이 필요하며, 중학교 수준의 수학 지식을 가지고 있어야 합니다. 또, 가능하면 git을 사용할 줄 알면 좋습니다. 

\newpage

\noindent \textbf{Tentative Course Outline}
\begin{center} 
\begin{flushleft}
\begin{itemize}

\item Week 0(before class) : Environment Settings / Python Reminder

\begin{itemize}
\item Python 개발 환경 설치 
\item 클래스 및 객제지향 : 문법, method, 상속 
\item 문제 풀이 : 피보나치 수 계산하기, 정렬, 소인수분해 
\item 소스 코드 작성, 저장, 실행 및 버젼관리 기초 : Git (\textit{Optional})
\end{itemize}

\item Week 1 : Math Basics/Implementation 

\begin{itemize} 
\item 수학 
\begin{itemize}
\item 집합(set)의 정의 및 연산 
\item 함수(function)의 정의
\item 유명한 함수들과 그 성질 살펴보기 : sin, cos, tan, log, exp
\end{itemize} 
\item 프로그래밍 
\begin{itemize} 
\item 객체지향적 사고 : 클래스를 이용한 집합 및 함수 구현
\item 재귀 : 피보나치 수열 계산으로 본 재귀함수 
\item 알고리즘의 시간복잡도 표기법 및 계산 
\end{itemize} 
\end{itemize}

\item Week 1.5 : Documentation/Version Control
\begin{itemize} 
\item \LaTeX 소개 
\item git 소개
\end{itemize} 

\item Week 2 : Linear Algebra(1)

\begin{itemize} 
\item 수학 
\begin{itemize}
\item 벡터와 행렬의 정의 및 연산 : 크기, 사칙연산, 행렬식
\item 직교좌표계 : 두 좌표간의 거리, 직선, 평면의 방정식
\item 선형독립과 span
\item 선형변환
\item 벡터와 함수 : 함수를 element로 가지는 벡터와, 벡터를 argument로 가지는 함수 
\end{itemize} 
\item 프로그래밍 
\begin{itemize} 
\item Overloading : 행렬/벡터의 연산 구현 
\item A bit of Functional Approach : functions as values 
\end{itemize} 
\end{itemize}


\item Week 2.5 : Support Vector Machine
\begin{itemize} 
\item Linearly Seperable Case 
\item non-linear case : Using Kernel Function
\item Example on text classification : Linear SVM model for MNIST
\end{itemize}


\item Week 3 : Linear Algebra(2) 

\begin{itemize} 
\item 수학 
\begin{itemize}
\item 행렬의 대각화 
\item 행렬의 고유값 찾기 
\item 행렬의 분해 : Singular Value Decomposition(SVD), QR Decomposition 
\end{itemize} 
\item 프로그래밍 
\begin{itemize} 
\item 2주차의 행렬 클래스를 확장하여 위 내용들 구현하기 
\item 선형방정식 solver 만들기 
\end{itemize} 
\end{itemize}

\item Week 3.5 : Various Applications of Matrix 

\begin{itemize} 
\item wavelet을 이용한 이미지 손실 압축
\item Principal Compnent Analysis(PCA) 구현해보기
\item 선형계획법/게임이론 (\textit{optional}) 
\end{itemize}

\item Week 4 : Calculus(1) - 미분

\begin{itemize} 
\item 수학 
\begin{itemize}
\item 함수의 극한 
\item 미분의 정의 및 구현 : Symbolic/Numeric
\end{itemize} 
\item 프로그래밍 
\begin{itemize} 
\item 수식 parser/계산기 구현 
\item Symbolic 미분기 구현 
\item Numeric 미분기 구현 
\end{itemize} 
\end{itemize}

\item Week 4.5 : 정규표현식
\begin{itemize} 
\item Parsing이란?
\item 정규표현식이란?
\item 정규표현식 파서 만들기
\end{itemize}


\item Week 5 : Calculus(2) - 적분

\begin{itemize} 
\item 수학 
\begin{itemize}
\item 적분의 정의 : 리만 합
\item 미분과의 관계 
\item 정적분/부정적분 : 다양한 함수의 적분 해보기  
\end{itemize} 
\item 프로그래밍 
\begin{itemize} 
\item Symbolic 적분기 구현 (부정적분, 정적분 모두에 대해서)
\item Numeric 적분기 구현 (정적분에 대해서)
\end{itemize} 
\end{itemize}

\item Week 5.5 : Fourier Transform 
\begin{itemize} 
\item Fourier Transform 이론 
\item Discrete Fourier Transform 이론 및 Fast Fourier Transform 구현 
\item FFT를 이용한 악기 소리 샘플 분석 
\end{itemize}

\item Week 6 : Calculus(3) - 다변수 미분

\begin{itemize} 
\item 수학 
\begin{itemize}
\item 편미분 
\item 그래디언트와 활용 
\item 라그랑주 승수법 
\end{itemize} 
\item 프로그래밍 
\begin{itemize} 
\item 편미분 구현 
\item 그래디언트 구현 
\end{itemize} 
\end{itemize}

\item Week 6.5 : Gradient Descent 
\begin{itemize} 
\item Gradient Descent 소개 
\item 다양한 Gradient Descent 알고리즘 구현 
\end{itemize}


\item Week 7 : Probability and Statistics

\begin{itemize} 
\item 수학 
\begin{itemize}
\item 확률의 정의  
\item 조건부확률과 베이즈 정리 
\item 확률변수와 확률분포함수
\item 정규분포
\item Central Limit Theorem (\textit{optional})
\item Bayesian Statistics 
\end{itemize} 
\item 프로그래밍 
\begin{itemize} 
\item 수치적분을 통한 확률값 계산 
\end{itemize} 
\end{itemize}

\item Week 7.5 : Feedforward Neural Net
\begin{itemize} 
\item Feedforward Neural Net 이론 소개
\item Backpropagration / Gradient Descent 
\item Feedforward Neural Net 구현
\end{itemize}


\item Week 8 : Introduction to Machine Learning - statistical classification 

\begin{itemize} 
\item 머신 러닝이란? 
\item 머신 러닝의 분류 : supervised/unsupervised, classification/regression/clustering
\item Classification 알고리즘들 소개 : Linear Regression, NBC, LDA
\end{itemize}


\end{itemize}
\end{flushleft}
\end{center}





%%%%%% THE END 
\end{document} 